\setpart{Tables arc-en-ciel}

	\setsection{Comparaison et avantages}

		Le problème des précédentes tables est que si deux chaînes entrent en collision, elles fusionnent. Il existe des méthodes pour que deux chaînes en collision ne fusionnent pas, les \glspl{rainbow}. Le principe : elles utilisent des fonctions de réductions différentes à chaque point de la chaîne.

		Les chances que deux chaînes fusionnent étaient de 1 avec la méthode classique\cite{Oech03}, avec les \glspl{rainbow} les chances de fusion sont de $\frac{1}{t}$. Le succès pour une table de taille $m*t$ :

		\begin{align*}
			P_{table} &= 1 - \overset{t}{\underset{i=1}{\Pi}}(1 - \frac{m_i}{N})
		\end{align*}

		Il est intéressant de noter que la probabilité de succès de $t$ tables classiques $m*t$ est la même qu'une \gls{rainbow} $mt*t$.

		\bigskip

		Ces tables offrent les mêmes avantages que les tables classiques sans certains inconvénients :
		\bi
			\item la recherche dans les tables est diminuée d'un facteur $t$.
			\item il est facile de générer des tables sans fusion mais elles comporteront alors forcément des collisions.
			\item les \glspl{rainbow} ont une taille constante tandis que les chaînes à points distincts ont une taille variable. Ce qui réduit considérablement les fausses alertes.
		\ei

	\setsection{Expérimentations}

		Le travail de Philippe Oechslin\cite{Oech03} nous permet d'obtenir des résultats expérimentaux comparant les tables à points distincts et les \glspl{rainbow}. Oechslin utilise comme test le crack de mots de passe du LanManager de MS Windows car il peut être effectué sur n'importe quel poste de travail standard.

		\bigskip

		En se basant sur les paramètres exprimés plus haut, il a choisi les valeurs suivantes pour les deux types de tables afin de comparer les résultats (après 500 mots de passe craqués) ;

		\bigskip

		\begin{centertab}{l|c|c}
			& Classique avec PC & Arc-en-ciel \\
			\hline
			$t$, $m$, $l$ & 4666, 8192, 4666 & 4666, 38 223 872, 1 \\
			\hline
			predicted coverage & 75.5\% & 77.5\%\\
			measured coverage & 75.8\% & 78.8\%\\
		\end{centertab}

		\bigskip

		Cette expérimentation montre que pour la même quantité de données, les \glspl{rainbow} ont la même probabilité de succès que les tables classiques avec points caractéristiques.

		\bigskip

		En comparant le temps de cryptanalyse, le nombre d'opérations, et le nombre de fausses alertes par cryptanalyses, la \gls{rainbow} est 7 fois plus rapides et contient 2.8 fois moins de fausses alertes, ce qui réduit aussi le nombre d'opérations par cryptanalyse de 7.1 fois. Cela se traduit par le fait que la taille des chaînes dans les tables classiques n'est pas constante.

		\bigskip

		On voit alors ressortir l'importance pour les chaînes d'être constante. Deux points ressortent concernant la constance des chaînes\cite{Oech03} :

		\bi
			\item Attraction fatale : la variation de la longueur des chaînes entraîne une variation de la probabilité de fusion. Cela pose problème avec les fausses alertes. Elles apparaîtront plus fréquemment dans des grandes chaînes de par la fusion.
			\item Frais plus important : une fausse alerte dans une grande chaîne va demander beaucoup de calculs, cette chaîne ayant une grande chance de fusionner avec une autre grande chaîne, ce qui augmente le coût pour vérifier la fausse alerte. De plus, lorsque la taille de la chaîne est inconnue, il faut regénérer toute la chaîne.
		\ei

	\setsection{Pistes d'améliorations}

		La méthode de calcul peut encore être améliorée. Pour les \glspl{rainbow}, la quantité de calculs augmente de façon quadratique de 1 à $\frac{t^2-1}{2}$. Pour les tables classiques, c'est une augmentation linéaire $t^2$.

		\bigskip

		Pour augmenter les chances de succès et la rapidité, il faut augmenter le nombre de \glspl{rainbow} et regarder les lignes de toutes les tables en même temps. En prenant l'exemple précédent avec 78\% de succès avec une table, avec 5 tables le succès passe à 99.9\%. Le résultat se trouvant dans les premières lignes, cela diminue le temps de recherche nécessaire.

		Ce n'est évidemment pas la seule amélioration possible.

\endinput{}
