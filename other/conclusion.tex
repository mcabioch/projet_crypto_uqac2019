\setpart{Conclusion}

	Nous avons vu l'introduction des \gls{TMTO} par Hellman en 1980, qui est une idée totalement novatrice. Puis, une nouvelle version de cette méthode proposé par Oeschlin en 2003 avec les \glspl{rainbow}. Entre-temps la méthode des points distincts, proposé par Rivest en 1982, a été améliorée à différentes reprises sans jamais être équivalente aux \glspl{rainbow}. Enfin nous avons découvert les points de contrôle et les tables parfaites, ainsi que l'équation permettant de comparer les différentes méthodes de compromis.

	\bigskip

	Il ressort de cette étude des différentes méthodes de \gls{TMTO} que pour l'instant la méthode des \glspl{rainbow} est la méthode offrant généralement la meilleure solution. Même si la recherche reste active, cette méthode semble être la meilleure que nous puissions obtenir sans exploiter la forme de la fonction à inverser.

	\bigskip

	Jusqu'à présent ces méthodes étaient surtout connues pour permettre d'inverser une fonction de hashage, et ainsi attaquer les bases de données où sont stockés les mots de passe hashés. Mais pour faire face à ce type d'attaque, un suffixe est ajouté aux mots de passe avant la phase de hashage, il s'agit du sel. Il faut alors reconstituer notre \gls{rainbow} pour chaque sel différent. En pratique, comme ces attaques sont réalisées sur des bases de données contenant des milliers de mots de passe, reconstruire une \gls{rainbow} représente un moindre coup par rapport au bénéfice.

	\bigskip

	Une autre utilisation des \gls{TMTO} qui pour l'instant ne semble pas trop utilisée est l'attaque par \gls{plaint} choisi pour inverser une clé de chiffrement tel que proposé par Hellman.

	\bigskip

	Au-delà, de leurs intérêts en attaque informatique, les \glspl{TMTO} sont également d'excellents outils à étudier d'un point de vue mathématique. Et c'est pour toutes ces raisons que nous pensons que ces méthodes ne vont pas disparaître d'ici peu. 

\endinput{}
