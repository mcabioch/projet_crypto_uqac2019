\setsection{Conclusion}

	Nous avons vue l'introduction des \gls{TMTO} par Hellman en 1980, qui est une idée totalement novatrice. Puis, une nouvelle version de cette méthode proposé par Oeschlin en 2003 avec les \glspl{rainbow}. Entre temps la méthode des points distincts, proposé par Rivest en 1982, a été améliorée à différents reprise sans jamais être équivalente aux \glspl{rainbow}. Enfin nous avons découvert les points de contrôles et les tables parfaites, ainsi que l'équation permettant de comparer les différentes méthodes de compromis.

	\bigskip

	Il ressort de cette étude des différentes méthode de \gls{TMTO} que pour l'instant la méthode des \glspl{rainbow} est la méthode offrant généralement la meilleure solution. Même si la recherche reste active, cette méthode semble être la meilleure que nous puissions obtenir sans exploiter la forme de la fonction à inverser.

	\bigskip

	Jusqu'à présent ces méthodes étaient surtout connues pour permettre d'inverser une fonction de hachage, et ainsi attaquer les bases de données où sont stockés les mots de passes hashés. Mais pour faire à ce type d'attaque, un suffixe est ajouté aux mots de passe avant la phase de hashage, il s'agit du sel. Il faut alors reconstitué notre \gls{rainbow} pour chaque sel différent. Ceci étant dit il peut être utile cela peut être intéressant car les bases de données attaqués contiennent généralement des milliers de mots de passes.
	\bigskip
	Une autre utilisation des \gls{TMTO} qui pour l'instant ne semble pas trop utilisé est l'attaque par mot en clair choisie pour inverser une clé de chiffrement tel que proposé par Hellman.
	\bigskip
	Au-delà, de leurs intérêts en attaque informatique, les \gls{TMTO} sont également d'excellent outils à étudier d'un point de vue mathématiques. Et c'est pour toutes ces raisons que nous pensons que ces méthodes ne vont pas disparaître d'ici peu. 

\endinput{}
