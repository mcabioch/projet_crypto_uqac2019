\setsection{Améliorations notables : fausses alertes et outils de mesure}
	Comme nous l'avons vue précédement (section 3) le problème de fusion de chaîne a amené a créé les tables arc-en-ciel. Mais il se peut aussi qu'il y ait collision entre la chaîne que nous générons et une des chaînes des tables. Lorsque ce cas arrive nous parlons de fausse alertes.
	De plus jusqu'à présent nous n'avons pas réellement présenté d'outils permettant de comparer les efficacités des différentes méthodes de compromis.

\setsubsection{Points de contrôles}
	Afin de vérifier que 2 chaînes ne fusionnent il a été proposé par \cite{checkpoints} d'instaurer des points de contrôles. Pour ce faire lors de la phase de la création de la table nous choisis


\setsubsection{Outils de comparaison}
	

\setsubsection{Conclusion}


\endinput{}
