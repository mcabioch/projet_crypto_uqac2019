\setsection{Variantes Méthode des Points Distincts}
	Comme nous l'avons vue précédement la méthode des points distincts introduites en 1982 \cite{Rivest}, consiste à fixer un critère d'arrêt. Ainsi, nous vérifions si le dernier élément de la chaîne se trouve dans une table seulement s'il respecte le même critère d'arrêt, limitant de ce fait le nombre de ces vérification.
	
	Cette technique est très utile pour améliorer les performances des tables de TMTO que nous qualifieront de classique, contrairement aux tablex arc-en-ciel. En effet comme nous l'avons vue ces dernières sont construite de façon à avoir une table de réduction différentes par colonne, il n'est donc pas aisé d'avoir des lignes de taille variable.

\endinput{}
