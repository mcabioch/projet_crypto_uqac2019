\setsection{Principes et origines}

	Dans son article \og{}A Cryptanalytic Time-Memory Trade-Off\fg{}\cite{ehellman} publié en juillet 1980, Martin \bsc{E. Hellman} décrit pour la première fois un compromis temps-mémoire. Cette technique est à la base de nombreuses méthodes de cryptanalyse.\\

	De nombreux problèmes comme le logarithme discret ou le problème du sac à dos permettent l'utilisation de ce compromis. Pour $N$ solutions à vérifier, le compromis temps-mémoire permet de trouver la solution en $T$ opérations avec $M$ mots de mémoire. On obtient alors le produit temps mémoire suivant : $TM = N$.

	La cryptanalyse est un problème de recherche permettant les deux extrêmes de recherches exhaustive ($T=N$, $M=1$) et ($T=1$, $M=N$). Cependant, avant E. Hellman\cite{ehellman}, aucun papier n'avait été publié sur un compromis entre les deux. De plus, avec des résultats de temps-mémoire tels que $T = M = N^{2/3}$ ou une compléxité de $M + T$, le compromis est effectivement plus efficace et rentable qu'une recherche exhaustive ou une recherche de table. Ainsi, casser un DES 56-bit avec cette méthode est moins complèxe que de casser un DES 38-bit avec une recherche exhaustive.

	Cependant, pour les problèmes cités plus haut, le compromis temps-mémoire n'est pas des plus efficace, des solutions existants avec $T = M = N^{1/2}$. Cela ne montre pas l'inefficacité du compromis mais que son amélioration est possible.\\

	Une recherche exhaustive peut se faire avec une attaque par \gls{plaint} connu, alors qu'une recherche par table nécessite une attaque par \gls{plaint} choisi.

	Dans une recherche exhaustive, le \gls{cipher} peut être déchiffré pour chaque clé et comparé au \gls{plaint}. Si les textes sont identiques, après quelques tests additionels rejetant les fausses alertes, la clé est trouvée.

	Dans une recherche par table, le cryptanalyste encrypte certains \gls{plaint} $P_0$ pour chaque clé $N$ possible. Ces $N$ \gls{cipher} sont enregistrés et triés dans des tables. Pour une nouvelle clé $K$ et le \gls{cipher} par cette clé $C_0 = S_K(P_0)$, le cryptanalyste retrouvera $C_0$ et la clé correspondante en $log_2 N$ opérations.\\

	La méthode du compromis temps-mémoire fonctionne avec ce type d'attaque en \gls{plaint} choisi. Cependant il fonctionne aussi avec une attaque par \gls{cipher} seulement.

	\clearpage
	II
	III\\

	\clearpage
	IV

\endinput{}
